%% LyX 2.0.3 created this file.  For more info, see http://www.lyx.org/.
%% Do not edit unless you really know what you are doing.
\documentclass[usenatbib]{mn2e}
\usepackage[latin9]{inputenc}
\usepackage[a4paper]{geometry}
\geometry{verbose}
\setcounter{tocdepth}{3}
\usepackage{color}
\usepackage{float}
\usepackage{graphicx}

\makeatletter

%%%%%%%%%%%%%%%%%%%%%%%%%%%%%% LyX specific LaTeX commands.
%% Because html converters don't know tabularnewline
\providecommand{\tabularnewline}{\\}
%% A simple dot to overcome graphicx limitations
\newcommand{\lyxdot}{.}


%%%%%%%%%%%%%%%%%%%%%%%%%%%%%% User specified LaTeX commands.






%%%%%%%%%%%%%%%%%%%%%%%%%%%%%% LyX specific LaTeX commands.
%% A simple dot to overcome graphicx limitations
%Make my life significantly easier

\global\long\def\bd{{\bm{\delta}}}

\makeatother

\begin{document}
We are interested in recovering the halos and their masses, positions
and velocities with the smallest time step necessary to preserve sufficient
properties as mock catalogs.


\section{Matching Halos}

First, we compare a halo by halo in different simulations. Since all
the simulations use exactly the same initial conditions, if those
approximated N-body simulations can recover the halos reasonably well
(compared to a full N-body simulation), then we should find the same
halo at the same position with the same mass in different samples.
The question here are how mass resolutions and time steps affect on
halo properties (i.e., mass, position and velocity) and how many halos
don't correspond to halos in different samples.

To spatially match the halos defined in different mass resolution
and time step parameters, we first find a pair of halos from two different
simulations, whose distance is the closest. Then, we set two conditions
on those pairs to declare that those paired halos are the same halos:

1) distance is smaller than $0.5h^{-1}{\rm Mpc}$,

2) mass ratio is smaller than $10^{0.5}{\rm M_{\odot}}$.

Under the above conditions, \textcolor{black}{more than}\textcolor{red}{{}
}90\% of halos in 450/5 found paired halos in 300/2 at redshift $z=0.15$
with a fixed mass resolution, $256^{3}$ particles (shown in \textcolor{black}{Table}\textcolor{red}{{}
}\ref{tab:unmatchedHalo}). We also checked how those conditions affect
to fraction of matched halos by changing the numerical values for
distance and mass ratio criteria. As shown in Table \ref{tab:unmatchedHalo},
changing a criterion on mass ratio did not change the fraction compared
to the distance criterion. This indicates that deviation of halo masses
on pairs is relatively small.

\begin{table*}[p]
\begin{tabular}{|c|c|c|c|}
\hline 
mass$[{\rm M_{\odot}]}$\textbackslash{}distance $[h^{-1}{\rm Mpc}]$ & 0.5 & 0.75 & 1.0\tabularnewline
\hline 
\hline 
$10^{0.5}$ & 0.0915 & 0.0842 & 0.0842\tabularnewline
\hline 
$10^{0.75}$ & 0.0909 & 0.0825 & 0.0817\tabularnewline
\hline 
$10^{1.0}$ & 0.0909 & 0.0823 & 0.0763\tabularnewline
\hline 
\end{tabular}

\caption{\label{tab:unmatchedHalo}Fractions of unmatched halos (over matched
halos) for the sample of 450/5 when we compare a halo by halo for
300/2 at redshift $z=0.15$. This table shows how the fraction is
changed according to changing the matching conditions for distance
and mass ratio.}


\end{table*}



\subsection{Mass Resolution}

In this section, we examine how mass resolutions affect to halo properties.
We use a $(256h^{-1}{\rm Mpc)^{3}}$-cubical box and three different
mass resolutions: $512^{3}$, $256^{3}$, and $128^{3}$ particles
with a fixed time steps (450/5). We take a simulation of $512^{3}$
particles as a reference. Table \ref{tab:unmatched_512} indicates
fractions of unmatched halos when we impose matching conditions. The
major reason that most halos in the sample of $512^{3}$ particles
can't find a matched halo in smaller mass resolution samples is because
total number of halos in each sample differ by an order of 10. In
other words, the total number of halo corresponds to the number of
particles in the box. -> \textcolor{red}{I wonder how FOF defines
halos in different mass resolution samples...}

\begin{table*}[p]
\begin{tabular}{|c|c|c|c|}
\hline 
mass-resolution\textbackslash{}z & 0.15 & 0.5 & 0.8\tabularnewline
\hline 
\hline 
$256^{3}$ & 0.872 & 0.879 & 0.889\tabularnewline
\hline 
$128^{3}$ & 0.997 & 0.998 & 0.998\tabularnewline
\hline 
\end{tabular}

\caption{\label{tab:unmatched_512}Fractions of unmatched halos (over matched
halos) for the sample of $512^{3}$ particles when we compare a halo
by halo for $256^{3}$ and $128^{3}$ particles with a fixed time
step (450/5). Most of halos are unmatched due to large differences
on numbers of halo in different mass resolution samples.}
\end{table*}


\begin{figure}
\includegraphics[width=1\columnwidth]{/Users/ts485/Dropbox/Tomomi/HACC/Conv/Matching_halos/plots_450_5/distance1_normed_450_5_z0\lyxdot 15} 

\caption{\label{fig:HaloPosition512}Histograms of distance difference for
matched halos with respect to halos from $512^{3}$ particles with
450 large steps and 5 inner steps. Different colors correspond to
different mass resolutions: $256^{3}$ particles and $128^{3}$ particles
with a fixed time step (450/5) at redshift $z=0.15$. }


\end{figure}


\begin{figure}
\includegraphics[width=1\columnwidth]{/Users/ts485/Dropbox/Tomomi/HACC/Conv/Matching_halos/plots_450_5/histogram3_normed_450_5_z0\lyxdot 15}

\caption{\label{fig:HaloMass512}Histograms of log-based mass difference for
different mass resolutions with respect to $512^{3}$ particles at
redshift $z=0.15$. Histograms are normalized and all simulations
have the same time step, 450/5. Colors correspond to $256^{3}$ particles
(blue) and $128^{3}$ particles (green).}


\end{figure}


\begin{figure}
\includegraphics[width=1\columnwidth]{/Users/ts485/Dropbox/Tomomi/HACC/Conv/Matching_halos/plots_450_5/velMag1_normed_450_5_z0\lyxdot 15}

\caption{\label{fig:HaloVelocity512}Histograms of velocity magnitude difference
for matched halos with respect to halos from $512^{3}$ particles
at redshift $z=0.15$. Different colors correspond to different mass
resolutions: $256^{3}$ particles (blue) and $128^{3}$ particles
(green). All the simulations use the same step size, 450/5. Note that
angle difference of velocity vectors for 95\% of matched halos in
the $256^{3}$-particle sample are within 0.3 radians and 90\% in
the $128^{3}$-particle sample are within 0.6 radians.}


\end{figure}


\begin{figure}
\includegraphics[width=1\columnwidth]{/Users/ts485/Dropbox/Tomomi/HACC/Conv/Matching_halos/plots_450_5/unmatchedHalo_256_450_5_z0\lyxdot 15}

\caption{\label{fig:unmatchedHalo512}Upper: Unmatched halo number density
for the simulation of 300 large steps and 2 inner steps matching with
450 large steps and 5 inner steps. Both are from $256^{3}$ particle
simulations. Lower: Ratio of unmatched halo number densities (which
are the same as the ones in the left plot) with respect to the corresponding
total number densities. Both plots are at redshift $z=0.15$. I am
not sure why there are more unmatched halos above $10^{14}{\rm M_{\odot}}$
for 150/2. -> I may check the conditions of surrounding halos...}


\end{figure}



\subsection{Time steps}

In this section, we examine how global steps and sub-cycles affect
to halo properties. The goal here is to know the smallest global steps
and sub-cycles required to preserve necessary properties as mocks. 

Here, all the samples have the same mass resolution, $256^{3}$ particles
in the box of $(256h^{-1}{\rm Mpc})^{3}$. We use 450/5 (450 global
steps and 5 sub-cycles) as a reference to other samples: 300/3, 300/2,
150/3, and 150/2.

\begin{figure*}
\includegraphics[width=1\columnwidth]{/Users/ts485/Dropbox/Tomomi/HACC/Conv/Matching_halos/Distance/distance_z0\lyxdot 15}

\caption{\label{fig:HaloPosition}Histograms of distance difference for matched
halos with respect to halos from $256^{3}$ particles with 450 large
steps and 5 inner steps. Different colors correspond to different
simulation stepsizes: 300 large steps with 3 inner steps (blue) and
2 inner steps (green), and 150 large steps with 3 inner steps (red)
and 2 inner steps (cyan). All the simulations use $256^{3}$ particles
at redshift $z=0.15$.}
\end{figure*}
 
\begin{table*}[H]
\begin{tabular}{|c|c|c|}
\hline 
z=0.15 & mean {[}$h^{-1}{\rm Mpc}${]} & std{[}$(h^{-1}{\rm Mpc)^{2}}${]}\tabularnewline
\hline 
\hline 
300/3 & 0.078 & 0.056\tabularnewline
\hline 
300/2 & 0.092 & 0.068\tabularnewline
\hline 
150/3 & 0.233 & 0.106\tabularnewline
\hline 
150/2 & 0.237 & 0.107\tabularnewline
\hline 
\end{tabular}%
\begin{tabular}{|c|c|c|}
\hline 
z=0.5 & mean {[}$h^{-1}{\rm Mpc}${]} & std{[}$(h^{-1}{\rm Mpc)^{2}}${]}\tabularnewline
\hline 
\hline 
300/3 & 0.078 & 0.053\tabularnewline
\hline 
300/2 & 0.089 & 0.062\tabularnewline
\hline 
150/3 & 0.196 & 0.095\tabularnewline
\hline 
150/2 & 0.202 & 0.097\tabularnewline
\hline 
\end{tabular}%
\begin{tabular}{|c|c|c|}
\hline 
z=0.8 & mean {[}$h^{-1}{\rm Mpc}${]} & std{[}$(h^{-1}{\rm Mpc)^{2}}${]}\tabularnewline
\hline 
\hline 
300/3 & 0.068 & 0.050\tabularnewline
\hline 
300/2 & 0.080 & 0.059\tabularnewline
\hline 
150/3 & 0.199 & 0.096\tabularnewline
\hline 
150/2 & 0.204 & 0.097\tabularnewline
\hline 
\end{tabular}

\caption{\label{tab:HaloPosition}Means and standard deviations for positional
differences in Figure \ref{fig:HaloPosition}.}


\end{table*}


Figure \ref{fig:HaloPosition} shows the halo position differences
of paired halos as distances. This histogram indicates that global
step has bigger effects on overall halo positions and different sub-cycles
have negligible effects. Most of halos for 300 global steps have their
center positions within 100kpc with respect to halo center positions
for 450/5, while the simulations for 150 global steps have more scatter
in the figure. Means and standard deviations for the histograms are
shown in Table \ref{tab:HaloPosition}.

\begin{figure*}
\includegraphics[width=1\columnwidth]{/Users/ts485/Dropbox/Tomomi/HACC/Conv/Matching_halos/Histogram/histogram3_diff_z0\lyxdot 15}

\caption{\label{fig:HaloMass}Histograms of log-based mass difference of different
simulation stepsizes with respect to the one with 450 large steps
and 5 inner steps. Histograms are not normalized and all halos are
from the simulations with $256^{3}$ particles. For each of simulations:
300 large steps with 3 inner steps (blue) and 2 inner steps (green),
and 150 large steps with 3 inner steps (red).}


\end{figure*}


\begin{table*}
\begin{tabular}{|c|c|c|}
\hline 
z=0.15 & mean & std\tabularnewline
\hline 
\hline 
300/3 & -0.005 & 0.067\tabularnewline
\hline 
300/2 & -0.011 & 0.075\tabularnewline
\hline 
150/3 & -0.035 & 0.074\tabularnewline
\hline 
150/2 & -0.059 & 0.087\tabularnewline
\hline 
\end{tabular}%
\begin{tabular}{|c|c|c|}
\hline 
z=0.5 & mean & std\tabularnewline
\hline 
\hline 
300/3 & -0.007 & 0.069\tabularnewline
\hline 
300/2 & -0.019 & 0.075\tabularnewline
\hline 
150/3 & -0.047 & 0.082\tabularnewline
\hline 
150/2 & -0.078 & 0.091\tabularnewline
\hline 
\end{tabular}%
\begin{tabular}{|c|c|c|}
\hline 
z=0.8 & mean & std\tabularnewline
\hline 
\hline 
300/3 & -0.009 & 0.069\tabularnewline
\hline 
300/2 & -0.026 & 0.077\tabularnewline
\hline 
150/3 & -0.065 & 0.084\tabularnewline
\hline 
150/2 & -0.100 & 0.094\tabularnewline
\hline 
\end{tabular}

\caption{\label{tab:HaloMass}Means and standard deviations for halo mass differences
in log-based halo mass (with base 10) in Figure \ref{fig:HaloMass}. }


\end{table*}


Figure \ref{fig:HaloMass} shows halo mass differences for paired
halos with respect to halos from 450/5. Global steps affect on overall
distribution properties (i.e., mean and standard deviation shown in
Table \ref{tab:HaloMass}), while sub-cycles affect on their amplitudes.
This is because sub-cycles changes small-scale dynamics and it can
cause a difference on number of halos declared through FOF, whose
linking length is fixed to $b=0.2$. In general, smaller step sizes
make distribution of DM particles as halos more diffused (since smaller
stepsizes means that dynamics is more linear(?)) and there is a possibility
that some of gathered DM particles are not considered as halos. In
\textcolor{red}{Manera et al.} which approximates N-body simulations
by using the second-order Lagrangian Perturbation Theory, they solved
this problem by changing the linking length for FOF. ->\textcolor{red}{{}
Can we tune (or do we need to tune) linking lengths based on step
sizes?}

\begin{figure*}[p]
\includegraphics[width=1\columnwidth]{/Users/ts485/Dropbox/Tomomi/HACC/Conv/Plots4lyx/velMag_z0\lyxdot 15}

\caption{\label{fig:HaloVelocity}Histograms of velocity magnitude difference
for matched halos with respect to halos from $256^{3}$ particles
with 450 large steps and 5 inner steps. Different colors correspond
to different simulation stepsizes: 300 large steps with 3 inner steps
(blue) and 2 inner steps (green), and 150 large steps with 3 inner
steps (red) and 2 inner steps (cyan). All the simulations use $256^{3}$
particles at redshift $z=0.15$. Note that angle difference of velocity
vectors for 98\% of matched halos are within 0.3 radians.}
\end{figure*}
 
\begin{table}
\begin{tabular}{|c|c|c|}
\hline 
z=0.15 & mean {[}${\rm km/{\rm s}}${]} & std {[}$({\rm km/{\rm s})^{2}}${]}\tabularnewline
\hline 
\hline 
300/3 & -3.54 & 23.27\tabularnewline
\hline 
300/2 & -3.90 & 25.65\tabularnewline
\hline 
150/3 & -17.77 & 28.83\tabularnewline
\hline 
150/2 & -17.94 & 29.30\tabularnewline
\hline 
\end{tabular}%
\begin{tabular}{|c|c|c|}
\hline 
z=0.5 & mean {[}${\rm km/{\rm s}}${]} & std {[}$({\rm km/{\rm s})^{2}}${]}\tabularnewline
\hline 
\hline 
300/3 & -4.36 & 33.23\tabularnewline
\hline 
300/2 & -4.07 & 34.54\tabularnewline
\hline 
150/3 & -25.30 & 40.72\tabularnewline
\hline 
150/2 & -26.34 & 42.26\tabularnewline
\hline 
\end{tabular}%
\begin{tabular}{|c|c|c|}
\hline 
z=0.8 & mean {[}${\rm km/{\rm s}}${]} & std {[}$({\rm km/{\rm s})^{2}}${]}\tabularnewline
\hline 
\hline 
300/3 & -6.37 & 39.96\tabularnewline
\hline 
300/2 & -6.66 & 42.77\tabularnewline
\hline 
150/3 & -26.11 & 51.25\tabularnewline
\hline 
150/2 & -27.83 & 54.28\tabularnewline
\hline 
\end{tabular}

\caption{\label{tab:HaloVelocity}Means and standard deviations for halo velocity
differences in Figure \ref{fig:HaloVelocity}.}


\end{table}


For halo velocities, the results are shown in Figure \ref{fig:HaloVelocity}.
The histogram is a function of velocity magnitude differences. For
150 global steps, the means slightly deviate from 0 and have negative
values. This means that magnitude of velocity for 150 global steps
is smaller than that for 450 global steps. One way to explain about
it is that because smaller time step halos are less dense, the potential
wells at center of halos may be less deeper than the ones for larger
step sizes. We also examined differences on velocity direction (orientation?).
More than 98\% of paired halos have angle differences (with respect
to halos for 450/5) within 0.3 radians. This implies that orientation
of velocities are well-preserved among the samples for different time
steps.

\begin{figure*}
\includegraphics[width=1\columnwidth]{/Users/ts485/Dropbox/Tomomi/HACC/Conv/Plots4lyx/New_unmatchedHalo_256_300_2_z0\lyxdot 15}\includegraphics[width=1\columnwidth]{/Users/ts485/Dropbox/Tomomi/HACC/Conv/Plots4lyx/New_unmatchedHaloRatio_256_z0\lyxdot 15}

\caption{\label{fig:unmatchedHalo}Upper: Unmatched halo number density for
the simulation of 300 large steps and 2 inner steps matching with
450 large steps and 5 inner steps. Both are from $256^{3}$ particle
simulations. Lower: Ratio of unmatched halo number densities (which
are the same as the ones in the left plot) with respect to the corresponding
total number densities. Both plots are at redshift $z=0.15$. I am
not sure why there are more unmatched halos above $10^{14}{\rm M_{\odot}}$
for 150/2. -> I may check the conditions of surrounding halos...}
\end{figure*}
 
\begin{table*}[p]
\begin{tabular}{|c|c|c|c|}
\hline 
time step/z & 0.15 & 0.5 & 0.8\tabularnewline
\hline 
\hline 
300/3 & 0.099 & 0.107 & 0.120\tabularnewline
\hline 
300/2 & 0.134 & 0.154 & 0.182\tabularnewline
\hline 
150/3 & 0.219 & 0.233 & 0.285\tabularnewline
\hline 
150/2 & 0.299 & 0.331 & 0.387\tabularnewline
\hline 
\end{tabular}

\caption{\label{tab:unmatched_256}Fractions of unmatched halos (over matched
halos) for the sample of 450/5 when we compare a halo by halo for
other time steps shown in the table with a fixed mass resolution ($256^{3}$
particles). }
\end{table*}


At last, we show number density of unmatched halos in Figure \ref{fig:unmatchedHalo}.
There are more unmatched halos for smaller halo masses shown in the
upper panel of Figure \ref{fig:unmatchedHalo}, which compares 450/5
and 300/2 samples. When we find a matched halo in two different samples,
an algorithm tries to find a matched halo in a smaller step size sample
(for the case of the upper panel of Figure \ref{fig:unmatchedHalo},
300/2) for a larger step size sample (i.e., 450/5), and yet the number
densities for unmatched halos for both samples are almost the same
(which I somehow feel a little bit strange...). Unmatched number densities
for different step sizes are shown in the lower panel of Figure \ref{fig:unmatchedHalo}.
As is clear, sub-cycles in the number of unmatched halos is negligible.
Again, changing the matching conditions don't affect on the fraction
of unmatched halos much and the fraction is always less than 10\%.

{*}{*}{*}snapshots for unmatched halos whose mass is greater than
$10^{14}{\rm M_{\odot}}$

There are two unmatched halos whose mass is greater than $10^{14}{\rm M_{\odot}}$
in 450/5. One of them was completely unmatched with any halos from
300/2 (shown in Figure \ref{unmatch1}), while the other was due to
the distance criteion on matching conditions (shown in Figure \ref{unmatch2}).

\begin{figure*}[t]
\includegraphics[width=1\columnwidth]{/Users/ts485/Dropbox/Tomomi/HACC/Conv/Matching_halos/Unmatched_revised/mass14_unmacthedHalo1_data}\includegraphics[width=1\columnwidth]{/Users/ts485/Dropbox/Tomomi/HACC/Conv/Matching_halos/Unmatched_revised/mass14_cont_unmatchedHalo1_sample}

\caption{\label{unmatch1}Two halos from 300/2 which are close to the unmatched
halo with halo mass $10^{14.003}{\rm M_{\odot}}$(from 450/5) are
separate more than $2h^{-1}{\rm Mpc}$ and are either larger ($10^{14.612}{\rm M_{\odot}})$
or smaller ($10^{13.3}{\rm M_{\odot}}$) than the mass criterion.}
\end{figure*}


\begin{figure*}[p]
\includegraphics[width=1\columnwidth]{/Users/ts485/Dropbox/Tomomi/HACC/Conv/Matching_halos/Unmatched_revised/mass14_unmacthedHalo2_data}\includegraphics[width=1\columnwidth]{/Users/ts485/Dropbox/Tomomi/HACC/Conv/Matching_halos/Unmatched_revised/mass14_cont_unmatchedHalo2_sample}

\caption{\label{unmatch2}This halo is considered as an unmatched halo, because
the closest halo from 300/2 has the distance of $0.55h^{-1}{\rm Mpc}$
(and log-mass difference of 0.1). The matching condition used here
was $0.5h^{-1}{\rm Mpc}$. }
\end{figure*}


Now, we checked the first halo more closely to investigate why there
is no declared halos...


\section{Observable/Statistics}
\begin{itemize}
\item Mass function comparisons: 
\item Halo power spectra and bias: 
\item Cross correlations with full N-body, higher particle loading etc 
\item Comparison of matched halo samples and mass-sliced samples: what the
comparison indicates is that corresponding halos don't have the same
masses. ->What kind of problems do we have by having this issue?->$b(M)$
may be diffrent for different simulations. 
\end{itemize}

\section{Observable Box}

What we want to check/know here are:

1) What redshift can we use linear-shifting?,

2) What redshift-step size is required to preserve dynamics in simulations?
\end{document}
