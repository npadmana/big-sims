\documentclass[usenatbib]{mn2e}
\usepackage{graphicx}
\usepackage{bm}
\usepackage{fixltx2e}
\usepackage{astrobib_mnras2e}
%\usepackage{lineno}

\begin{document}
\topmargin-1cm
%\linenumbers

%Make my life significantly easier
\newcommand{\bd}{{\bm \delta}}


\title[Mock catalogs for BOSS]
{Generating Mock Catalogs for the Baryon Oscillation Spectroscopic Survey : An Approximate N-Body approach}
\author[People??]{Aaronson Aardvark$^{1}$ \\
$^{1}$ Universe \\
}

\date{\today}
\maketitle

\begin{abstract}
We introduce and test an approximate scheme for generating mock catalogs for
large-scale structure measurements in galaxy surveys, specializing in this work
to the Baryon Oscillation Spectroscopic Survey.
A brief description of the approximation scheme goes here??? A brief description
of the tests and the accuracy we reach goes here??? Some comments about the
timings of the tests goes here??? Final comments about the BOSS samples go
here????
\end{abstract}

\section{Introduction}

Why do we need large numbers of large N-body simulations? Discuss both
covariance matrix estimation as well as systematic error estimation. Emphasize
importance to capture as much of the physics as is possible.

What about just running large numbers of N-body simulations : surveys are
getting larger, and the numbers of simulations required is also large
for covariance matrices. Ideally also require variations in cosmological
parameters. Mention the LasDamas suite of simulations here as exemplars of this
approach. 

Other approaches discussed here : Gaussian and log-normal random simulations;
issues with these. 

2LPT, PTHalos, Pinocchio : discuss recent work by Manera et al. Also reference
work in progress by Tassev et al.

Our approach -- tune down N-body simulations.

A brief discussion of BOSS here and the need for large N-body simulations there. 

An outline of the paper needs to go here\ldots

\section{Generating Approximate Simulations}

A brief discussion of the HACC code goes here, pointing out all the great things
it can do. 

Focus in on the key aspects of the code that we use for the approximate scheme :
time steps on the PM side, and subcycles. Explain the parameters we tune.


\section{Convergence and Timing Tests}

Describe the suites of simulations we run for convergence tests. 

(a) 256 Mpc boxes. How many do we need?  Which
approximation parameters should we choose to vary? What are our references --
full N-body with same particle loading, full N-body with higher particle
loading? Emulator?

(b) do we need any other box sizes? What do we hope to learn from these?

\subsection{Dark Matter}

What tests do we need here? A minimum list I can think of 
\begin{itemize}
  \item Convergence of the power spectrum as a function of approximation and
  redshift.
  \item Cross-correlation between full and approximate N-body.
\end{itemize}

\subsection{Halos}

\begin{itemize}
  \item Mass function comparisons
  \item Halo position, mass and velocity comparisons.
  \item Halo power spectra and bias. 
  \item Cross correlations with full N-body, higher particle loading etc
\end{itemize}

\section{The BOSS Simulations}

\subsection{Simulation Parameters}

describe box size, masses, geometry etc. Show that we can fit in two BOSS
volumes per box. 

\subsection{Building the Galaxy Catalogs}

How to do the redshift evolution -- interpolating between different redshift
snapshots

HODs at least for the basic BOSS sims. 

\subsection{An Application}

Maybe something simple here.

\section{Discussion}

Some words on conclusions go here. 


\end{document}




